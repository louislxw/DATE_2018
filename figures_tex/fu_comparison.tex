\pgfplotsset{
	axis background/.style={fill=white},
	tick style=black,
	tick label style=black,
	grid=both,
	xtick pos=left,
	ytick pos=left,
	tick style={
		major grid style={style=white,line width=1pt},minor grid style=white,
		tick align=outside,
	},
	minor tick num=4,
}

\begin{figure}[tb]
%\vspace{-18pt}
\centering
%\pgfplotstableread{
%	0  5			7     
%	1  12      		9     
%	2  8	     	6      
%	3  22    		8 
%	4  17    		9      
%	5  30     		11    
%	6  28     		13       
%	7  19      		11
%}\dataset
\pgfplotstableread{
	0  		 5		7     8
	1  		 12     9     8
	2  		 8	    6     8 
	3  		 22    	8 	  8
	4  	 	 17    	9     8
	5  		 30     11    8
	6  		 28     13    8
	7  		 19     11    8
}\dataset
\begin{tikzpicture}
\centering
\begin{axis}[ybar=0pt,
%enlarge x limits=0.05,
width=17cm,
x = 0.85cm,
height=4.3cm,
ymin=0,
ymax=100,        
ylabel={Number of FUs required},
grid style={dotted,gray},
ymajorgrids=true,
nodes near coords,    
xtick=data,
bar width = 0.25,
%xticklabels={1,2,...,26},
xticklabels = {
	\strut 1,
	\strut 2,
	\strut 3,
	\strut 4,
	\strut 5,
	\strut 6,
	\strut 7,
	\strut 8
%	\strut 26 (1)                                    
},
x tick label style={rotate=45, anchor=north east, inner sep=0mm},
major x tick style = {opacity=0},
minor x tick num = 1,
minor tick length=1ex,
every node near coord/.append style={
        anchor=west,
        rotate=90,
        font=\tiny
},
]

\addplot[draw=black,fill=blue!80, draw opacity=1] table[x index=0,y index=1] 
\dataset; \label{scoverlay}
\addplot[draw=black,fill=black!70, draw opacity=1] table[x index=0,y index=2] \dataset;\label{tmoverlay_1} 
\addplot[draw=black,fill=green!60, draw opacity=1] table[x index=0,y index=3] \dataset;\label{tmoverlay_2} 

\end{axis}
	\node [draw=none, fill=white] at (rel axis cs: 0.5,0.72) {\shortstack[l]{
			\ref{scoverlay} Spatially Configured Overlay~\cite{jain2015efficient} \\ \ref{tmoverlay_1} V1/V2 \\ \ref{tmoverlay_2} V4}};

\end{tikzpicture}
\caption{Number of FUs required for the benchmarks.}
\label{fu_comparison}
%\vspace{-9pt}
\end{figure}