\lstset { %
	language=C,
	backgroundcolor=\color{white}, % set backgroundcolor
	%basicstyle=\footnotesize,% basic font setting
	%basicstyle=\ttfamily\tiny,
	basicstyle=\ttfamily\tiny,
	keywordstyle=\color{blue}\ttfamily,
	stringstyle=\color{red}\ttfamily,
	commentstyle=\color{green}\ttfamily,
	breaklines=true	
}
\lstset{framesep=-10pt, xleftmargin=-10pt}

\begin{table}[!h]
	\centering
	\caption{DFG Description of the compute kernel}
	\label{code.tbl}
	\begin{tabular}{l}
		%    \toprule
%		\multicolumn{1}{c}{(a) C description}  &\multicolumn{1}{c}{(b) DFG description} \\ % Assembly with Loopback Optimization
		%    \midrule
		%    \hspace{-0.2in}
		%    \hspace{-0.2in}
		\begin{lstlisting}
digraph graphname {
N1 [ntype="invar", label="I0_N1"];
N2 [ntype="invar", label="I1_N2"];
N3 [ntype="invar", label="I2_N3"];
N4 [ntype="invar", label="I3_N4"];
N5 [ntype="invar", label="I4_N5"];
N6 [ntype="operation", label="sub_N6"];
N7 [ntype="operation", label="sub_N7"];
N8 [ntype="operation", label="sub_N8"];
N9 [ntype="operation", label="sub_N9"];
N10 [ntype="operation", label="sqr_N10"];
N11 [ntype="operation", label="sqr_N11"];
N12 [ntype="operation", label="sqr_N12"];
N13 [ntype="operation", label="sqr_N13"];
N14 [ntype="operation", label="add_N14"];
N15 [ntype="operation", label="add_N15"];
N16 [ntype="operation", label="add_N16"];
N17 [ntype="outvar", label="O0_N17"];
N1 -> N6;
N2 -> N7;
N3 -> N6;
N3 -> N7;
N3 -> N8;
N3 -> N9;
N4 -> N8;
N5 -> N9;
N6 -> N10;
N7 -> N11;
N8 -> N12;
N9 -> N13;
N10 -> N14;
N11 -> N14;
N12 -> N15;
N13 -> N15;
N14 -> N16;
N15 -> N16;
N16 -> N17;
}
		\end{lstlisting}\\
		%    \bottomrule
	\end{tabular}
\end{table}
