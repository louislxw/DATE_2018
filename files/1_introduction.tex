%!TEX root=../draft.tex
%\vspace{-6pt}
\section{Introduction}
\label{ch1_introduction}

There has been a resurgence in FPGA-based accelerators due to developments in the cloud computing and IoT domains. 
FPGA accelerators are often custom designed to achieve maximum performance, using conventional RTL hardware design techniques, and as such, are only applied to specific algorithms in specific applications, and hence are fixed, negating many of the benefits of the programmable FPGA device. This design process is long and complex, requiring low-level device expertise and special knowledge of both hardware and software systems, resulting in major design productivity issues and long compilation times, which limit the mainstream adoption of FPGA based accelerators for general purpose computing. 
%for general purpose computing~\cite{stitt2011field}.

High-level synthesis (HLS) has been widely adopted by EDA tool vendors to address the design productivity issue. However, to maximize performance, detailed low-level design effort is often still required, making design difficult for non-experts. Additionally, while HLS tools have allowed designers to focus on high-level functionality instead of low-level details, the back-end flow (specifically the FPGA place and route) requires very long compilation times, particularly for large designs, contributing to the lack of productivity and mainstream adoption of FPGAs. In many cases, the time required to change an FPGA configuration limits hardware accelerators to predesigned static (i.e. fixed) implementations, negating the fundamental benefit of FPGAs. To be more appealing to the broader group of application developers, who are used to software API abstractions and fast development cycles, the FPGA hardware resource needs to be better abstracted.

One possible solution to this problem is to use an overlay (a programmable coarse-grained hardware abstraction layer on top of the FPGA fabric) as this simplifies both the hardware design and mapping process. This then allows the FPGA to be treated as a virtualized execution platform that both abstracts the hardware details and enables runtime management support, so that the hardware can be viewed as just another software-managed task, possibly even controlled by the OS or a hypervisor~\cite{jain2014virtualized}. This results in better application management, and has the potential for allowing portability across devices, software-like programmability by mapping from high-level descriptions, better design reuse, fast compilation by avoiding the complex FPGA design flow (particularly the very slow place and route process), resulting in improved design productivity. Another significant advantage is rapid application swapping (a hardware context switch) as coarse-grained overlay architectures have smaller configuration data sizes than fine-grained FPGAs.  The major problem is that many of the current overlays are not efficient (in area, power, throughput, etc.) and still require FPGA-like configuration times, as in many cases the overlay needs to change when the application requiring acceleration needs to change.
